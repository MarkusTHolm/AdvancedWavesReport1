\chapter{Implementation of the method for unidirectional waves}
The JONSWAP spectrum is defined as
\begin{subequations}
\begin{equation}
    S_{\eta \eta}(f) = \alpha H_s^2 f_p^4 f^{-5} \gamma^\beta \exp{\left[-\frac{5}{4}\left(\frac{f_p}{f}\right)^4\right]}
    \label{eq:JONSWAPSpec}
\end{equation} 
\begin{equation}
    \alpha \approx \frac{0.0624}{0.230+0.0336\gamma-\frac{0.185}{1.9+\gamma}} ,\quad \beta = \exp{\left[ -\frac{(f-f_p)^2}{2\sigma^2f_p^2} \right]}, \quad
    \sigma =
    \begin{cases}
    0.07, \quad \text{for }f\leq f_p \\
    0.09, \quad \text{for }f> f_p
    \end{cases}
    \label{eq:JONSWAPparam}
\end{equation}
\end{subequations}
where \cref{eq:JONSWAPSpec} defines the power spectrum and \cref{eq:JONSWAPparam} defines the suggested scalar parameters given in the slides. Here, $H_s$ is the significant wave height, $f_p$ is the frequency corresponding to the peak period (i.e. $f_p=1/T_p$), $f$ is the frequency of interest, $\gamma$ is the so-called peak enchancement factor and $\alpha$, $\beta$, $\sigma$ are scalar parameters describing the shape of the JONSWAP spectrum.

\begin{figure}[h!]
\begin{subfigure}[t]{.9\textwidth}
\centering
\begin{tikzpicture}
\pgfplotsset{set layers}
\begin{axis}[
	width = .50\linewidth,
	height = .35\linewidth,
	xmin = 0, xmax = 0.016,
	ymin = -5, ymax = 25,
	grid = both,
	scale only axis,
    axis y line*=left,
	xlabel = Square of element size: $h^2$ \si{[m^2]},
	ylabel style={align=center}, ylabel= Pressure: $p$ \si{[Pa]} or \\ Energy dissipation: $\Phi$ \si{[W/m]},
	minor tick num = 0,
	major grid style = {lightgray},
	minor grid style = {lightgray!25},
	legend cell align = {left},
	legend pos = south west,
    cycle list name=LineStyles,
    every x tick scale label/.style={
    at={(.9,0)},yshift=-28pt,anchor=south west,inner sep=0pt},
]
\addplot [black, solid, mark = *,mark options={fill=white, scale=0.7}] table {Figures/PlotData/convpoisback_pa.dat};
\addplot [black, solid, mark = triangle*,mark options={fill=white, scale=0.9}] table {Figures/PlotData/convpoisback_pb.dat};
\addplot [black, dashed, mark = *,mark options={solid, fill=white, scale=0.7}] table {Figures/PlotData/convpoisback_phi.dat};
\legend{
	{\footnotesize $p_\text{A}(h^2)$},
	{\footnotesize $p_\text{B}(h^2)$},
	 {\footnotesize $\Phi(h^2)$},
	 }
\end{axis}
\begin{axis}[
	width = .50\linewidth,
	height = .35\linewidth,
    scale only axis,
	xmin = 0, xmax = 0.016,
    ymin = 0.4, ymax = .6,
	axis y line* =right,
	axis x line = none,
	legend pos = south east,
	ylabel = Horizontal velocity: $u_x$ \si{[m/s]}
]
\addplot [black, dotted, mark = *,mark options={solid, fill=white, scale=0.7}] table {Figures/PlotData/convpoisback_uxa.dat};
\addplot [black, dotted, mark = triangle*,mark options={solid, fill=white, scale=0.9}] table {Figures/PlotData/convpoisback_uxb.dat};
\legend{
     {\footnotesize $u_{x,\text{A}}(h^2)$},
     {\footnotesize $u_{x,\text{B}}(h^2)$},
}
\end{axis}
\end{tikzpicture}
\caption{Quantitative convergence analysis of pressure $p$, horizontal velocity, $u_x$, and energy dissipation, $\Phi$. }
\label{fig:convpoisbackfig}
\end{subfigure}
\end{figure}
